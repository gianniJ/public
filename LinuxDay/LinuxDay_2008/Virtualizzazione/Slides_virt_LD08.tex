\documentclass[ignorenonframetext]{beamer}
\usepackage{qtree}
\usepackage{eso-pic}
\usepackage{hyphenat}

\usepackage[italian]{babel}
\usepackage[all]{xy}
\usepackage{amsmath}
\usepackage{amsfonts}
\usepackage{amssymb}
%\usepackage{ctable}

%\usepackage{listings}

\usepackage{graphicx}
\usepackage{fontspec,xltxtra,xunicode}
\defaultfontfeatures{Mapping=tex-text}

\setmainfont{Times New Roman}
\setsansfont{Arial}
\setmonofont{Courier New}

\usetheme{Berlin}
%\lstloadlanguages{Java}
%\lstset{language=Java,tabsize=1,basicstyle=\small,breaklines=true,numbers=left,numberstyle=\tiny,frame=tlrb}
%%% \lstinputlisting[title=Interval.java]{/home/the_nihilant/srcs/java/midterm_pa/Interval.java}


\title{Il mondo della virtualizzazione}
\date{\today}
\author{Claudio Imbrenda}
\institute{Linux Day 2008 PISA}

%\newcommand{\mq}[1]{\langle#1\rangle}
%\newcommand{\uberarrow}[1]{\xymatrix{\ar@{|->}[r]^{#1}&}}
%\newcommand{\uberarrowex}[1]{\xymatrix@C=70pt{\ar@{|->}[r]^{#1}&}}

%\newcommand{\seipt}{\fontsize{6}{0}\selectfont}

%\newcommand{\todo}[1]{\textcolor{red}{\textbf{TODO:} #1}}

%\newcommand{\EUR}{\textgreek{\euro}}

\begin{document}

  \begin{frame}
  \titlepage
  \end{frame}
  
  \begin{frame}
  \frametitle{Perché}
  Può essere utile avere una macchina virtuale:
  \begin{enumerate}
  \item Software scritto per un altro tipo di processore
  \item Software scritto per un altro sistema operativo
  \item Necessità di eseguire più sistemi operativi sulla stessa macchina
  \item Necessità di separare diverse istanze dello stesso sistema operativo
  \item Debug del sistema
  \end{enumerate}
  \end{frame}
  
  \begin{frame}
  \frametitle{Come}
  Gli approcci esistenti:
  \begin{itemize}
  \item interpretazione
  \item ricompilazione dinamica
  \end{itemize}
  \end{frame}
  
  \begin{frame}
  \frametitle{Una domanda viene spontanea...}
  Perché mai dovrei \textit{interpretare} o \textit{ricompilare al volo} un
  se ho un processore che è capace di eseguire nativamente?
  \end{frame}
  
  \begin{frame}
  \frametitle{Niente di nuovo, in realtà...}
  
  La \textbf{virtualizzazione} è esattamente questo: esecuzione nativa da parte del
  processore del sistema virtuale. 
  
  Introdotta negli anni '70 dalla IBM!  
  \end{frame}
  
  \begin{frame}
  \frametitle{Alcune proprietà interessanti}
  Virtualizzazione significa quindi che:
  
  \begin{itemize}
  \item Il sistema eseguito nell'ambiente virtuale si deve comportare
  esattamente come se fosse eseguito su una macchina fisica eqivalente
  \item L'ambiente di virtualizzazione deve avere il completo controllo delle risorse
  virtualizzate
  \item Una percentuale statisticamente grande di istruzioni devono essere
  eseguite senza intervento dell'ambiente di virtualizzazione.
  \end{itemize}
  
  L'ultima proprietà non è in realtà fondamentale, essa in fatti garantisce
  solo l'\textit{efficienza} della macchina virtuale.
  
  \end{frame}
  
  \begin{frame}
  \frametitle{Requisiti hardware}
  I requisiti di Popek e Goldberg sono i requisiti che una macchina deve
  avere per poter essere virtualizzabile. Per definire i requisiti per la
  virtualizzazione Popek e Goldberg introducono una classificazione delle
  istruzioni in tre gruppi:
  \begin{description}
  \item[Istruzioni privilegiate] Sono quelle che generano una \textit{trap} o
  un \textit{fault} quando sono eseguite in modalità utente e che invece
  vengono eseguite correttamente in modalità supervisore.
  \item[Istruzioni sensibili di controllo] Sono quelle che alterano lo stato
  globale del sistema.
  \item[Istruzioni sensibili di compotamento] Sono quelle il cui risultato
  dipende dallo stato globale del sistema.
  \end{description}
  \end{frame}
  
  \begin{frame}
  \frametitle{I teoremi di Popek e Goldberg}
  I risultati principali di Popek e Goldberg sono quindi:
  
  \begin{theorem}
  È possibile virtualizzare un'architettura se tutte le istruzioni sensibili
  sono privilegiate. 
  \end{theorem}
  
  \begin{theorem}
  Un'architettura è ricorsivamente virtualizzabile se:
  \begin{enumerate}
  \item è virtualizzabile
  \item è possibile costruire per essa una macchina virtuale che non abbia
  dipendenze di temporizzazione
  \end{enumerate}
  \end{theorem}
  \end{frame}
  
  \begin{frame}
  \frametitle{Che fine ha fatto la virtualizzazione?}
  Dopo gli anni 70 i computer sono diventati molto più comuni, ed economici.
  I principali motivi che spingevano la virtualizzazione sono venuti a
  mancare, fino alla fine degli anni '90.
  
  \begin{enumerate}
  \item Software scritto per un altro tipo di processore
  \item Software scritto per un altro sistema operativo
  \item Necessità di eseguire più sistemi operativi sulla stessa macchina
  \item Necessità di separare diverse istanze dello stesso sistema operativo
  \item Debug del sistema
  \end{enumerate}
  \end{frame}
  
  \begin{frame}
  \frametitle{Ma c'è un piccolo particolare...}
  L'architettura purtroppo più diffusa (x86) ..
  
  \textbf{non è virtualizzabile!}
  \end{frame}
  
  \begin{frame}
  \frametitle{Si trova sempre un modo..}
  \begin{itemize}
  \item Spesso i sistemi operativi non dipendono dalle istruzioni sensibili. 
  \item Spesso si può prevedere quando verranno usate istruzioni sensibili.
  \item Spesso si può barare!
  \end{itemize}
  \end{frame}
  
  \begin{frame}
  \frametitle{Esecuzione in modo utente}
  Una tecnica di virtualizzazione prevede che tutto il sistema virtualizzato
  venga eseguito in modo utente, in modo che le istruzioni sensibili
  privilegiate generino delle trap, sperando che non vengano usate le
  istruzioni sensibili non privilegiate. L'ambiente di virtualizzazione si
  occupa di gestire le trap per cercare di garantire la proprietà di equivalenza.
  
  Questo garantisce la massima velocità ma non certo la massima generalità.
  \end{frame}
  
  \begin{frame}
  \frametitle{Esecuzione in modo utente con patching al volo}
  Come nel caso precedente, con la differenza che il codice viene analizzato
  e modificato prima di essere eseguito in modo da mantenere la proprietà di
  equivalenza. 
  
  \begin{itemize}
  \item Richiede di conoscere dove e quando effettuare il patching.
  \item Ogni sistema da virtualizzare va trattato come caso a parte.
  \item Permette di virtualizzare anche sistemi che normalmente non
  sarebbero virtualizzabili.
  \end{itemize}
  \end{frame}
  
  \begin{frame}
  \frametitle{Barare spudoratamente!}
  Possiamo \textbf{modificare il set di istruzioni}, in modo che
  tutte le istruzioni sensibili diventino privilegiate. Intel e AMD hanno
  fatto esattamente questo con le loro estensioni di virtualizzazione.
  
  In questo modo si riesce a garantire sempre la proprietà di equivalenza.
  \end{frame}
  
  \begin{frame}
  \frametitle{Barare ancora più spudoratamente!}
  Infine possiamo modificare il sistema da virtualizzare in modo che si
  adatti in modo specifico al sistema di virtualizzazione.
  
  \textbf{In questo modo si viola la proprietà di equivalenza!}
  
  {\scriptsize (ma si raggiungono ottime performance!)}
  
  In questo caso si parla di paravirtualizzazione.
  \end{frame}
  
  \begin{frame}
  \frametitle{Qualche esempio concreto}
  \begin{itemize}
  \item Bochs (interprete)
  \item BasiliskII (interprete/ricompilatore)
  \item SheepShaver (interprete/ricompilatore)
  \item PearPC (interprete/ricompilatore)
  \item QEMU (ricompilatore/virtualizzatore)
  \item KVM (virtualizzatore/ricompilatore)
  \item VirtualBox (virtualizzatore/ricompilatore)
  \item Xen (virtualizzatore/paravirtualizzatore)
  \end{itemize}
  \end{frame}
  
  \begin{frame}
  \frametitle{Xen}
  \begin{itemize}
  \item necessita sempre di un sitema paravirtualizzato per accedere
  all'hardware (domain0)
  \item necessita delle estensioni di virtualizzazione per effettuare
  virtualizzazione
  \end{itemize} 
  \end{frame}
  
  \begin{frame}
  \frametitle{QEMU}
  \begin{itemize}
  \item inizialmente era solo un emulatore
  \item aggiunta in seguito la possibilità di virtualizzare\\ (modulo KQEMU)
  \item non fa uso di estensioni hardware per la virtualizzazione
  \end{itemize}
  \end{frame}

  \begin{frame}
  \frametitle{KVM}
  \begin{itemize}
  \item deriva da QEMU
  \item necessita delle estensioni di virtualizzazione
  \item la componente kernel-space è parte del kernel di linux
  \end{itemize}
  \end{frame}
  
  \begin{frame}
  \frametitle{VirtualBox}
  \begin{itemize}
  \item effettua esecuzione diretta quando possibile
  \item effettua ricompilazione al volo
  \item e modifica il codice durante le ricompilazioni, per non doverle
  rieffettuare
  \end{itemize}
  \end{frame}
  
  \begin{frame}
  \Huge Dimostrazione pratica\\
  (VirtualBox)
  \end{frame}

  \begin{frame}
  \Huge Dimostrazione pratica\\
  (QEMU)  
  \end{frame}

  \begin{frame}
  \frametitle{Domande?}
  
  \end{frame}

  \begin{frame}
  \frametitle{Fine}
  Happy hacking!
  \end{frame}

\end{document}
