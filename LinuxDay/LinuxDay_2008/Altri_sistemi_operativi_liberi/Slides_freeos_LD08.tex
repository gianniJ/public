\documentclass{beamer}
%\documentclass[ignorenonframetext]{beamer}
\usepackage{qtree}
\usepackage{eso-pic}
\usepackage{hyphenat}

\usepackage[greek,italian]{babel}
\usepackage[all]{xy}
\usepackage{amsmath}
\usepackage{amsfonts}
\usepackage{amssymb}
%\usepackage{ctable}

%\usepackage{listings}

\usepackage{graphicx}
\usepackage{fontspec,xltxtra,xunicode}
\defaultfontfeatures{Mapping=tex-text}

\setmainfont{Times New Roman}
\setsansfont{Arial}
\setmonofont{Courier New}

\usetheme{Warsaw}
%\lstloadlanguages{Java}
%\lstset{language=Java,tabsize=1,basicstyle=\small,breaklines=true,numbers=left,numberstyle=\tiny,frame=tlrb}
%%% \lstinputlisting[title=Interval.java]{/home/the_nihilant/srcs/java/midterm_pa/Interval.java}


\title{Altri sistemi operativi liberi}
\date{\today}
\author{Claudio Imbrenda}
\institute{Linux Day 2008 PISA}

%\newcommand{\mq}[1]{\langle#1\rangle}
%\newcommand{\uberarrow}[1]{\xymatrix{\ar@{|->}[r]^{#1}&}}
%\newcommand{\uberarrowex}[1]{\xymatrix@C=70pt{\ar@{|->}[r]^{#1}&}}

%\newcommand{\seipt}{\fontsize{6}{0}\selectfont}

\newcommand{\todo}[1]{\textcolor{red}{\textbf{TODO:} #1}}

%\newcommand{\EUR}{\textgreek{\euro}}

\newcommand{\freeos}[7]{
\begin{frame}
\frametitle{#1 (dettagli)}
\begin{tabular}{rp{8cm}}
\includegraphics[width=3cm]{#1-logo.jpg}
&\textbf{#1} (ultima versione: #2)\\
\hline
Licensa:&#3\\
Tipo:&#4\\
Kernel:&#5\\
Architetture:&#6\\
Sito di riferimento:&#7\\
\end{tabular}
\end{frame}
\begin{frame}
\frametitle{#1 (screenshot)}
\includegraphics[width=9.5cm]{#1-screenshot.jpg}
\end{frame}
}

\begin{document}
  \begin{frame}
  \titlepage
  \end{frame}
  
  \begin{frame}
  \frametitle{Alcuni concetti base}
  
  Qualche chiarimento sulla terminologia:
  
  \begin{description}
  \item[Tipo] Interfaccia delle chiamate di sistema.
  \item[Kernel] Tipo di kernel: generalmente monolitico o ibrido (vanno
  tanto di moda!)
  \item[Architetture] Le architetture hardware supportate dal sistema
  operativo.
  \end{description}
  \end{frame}
  
  \freeos{Linux}{2.6.27}{GPL v.2}{Unix-like}{Monolitico modulare}
  {alpha, arm, avr32, cris, h8/300, hp-pa, ia64, m32r, m68k, mips, powerpc,
  s/390, system z, sparc, superH, v850, x86, xtensa}{www.kernel.org}
  
  \freeos{Minix}{3.1.3}{BSD}{Unix-like}{Microkernel}
  {i386, \textit{arm}, \textit{powerpc}, \textit{mips}}{www.minix3.org}
  
  \freeos{OpenSolaris}{2008.05}{CDDL}{Unix-like}{Monolitico modulare}
  {sparc, x86, system z, \textit{powerpc}}{www.opensolaris.com}
  
  \freeos{Haiku}{alpha-r28303}{MIT}{BeOS}{Ibrido}{i386}{www.haiku-os.org}
  
  \freeos{ReactOS}{0.3.6}{GPL, LGPL, BSD}{Windows}{Ibrido}{i386}{www.reactos.org}
  
  \freeos{FreeBSD}{7.0}{BSD}{Unix-like}{Monolitico modulare}
  {x86, ia64, powerpc, sparc64}{www.freebsd.org}
  
  \freeos{OpenBSD}{4.3}{BSD}{Unix-like}{Monolitico modulare}
  {alpha, arm, hp-pa, m68k, m88k, mips, powerpc, superH, sparc, vax, x86}
  {www.openbsd.org}
  
  \freeos{NetBSD}{4.0.1}{BSD}{Unix-like}{Monolitico modulare}
  {alpha, arm, hp-pa, m68k, mips, powerpc, superH, sparc, vax, x86, \textit{ia64}}
  {www.netbsd.org}
  
  \freeos{DragonFlyBSD}{2.0.1}{BSD}{Unix-like}{Ibrido}{i386}{www.dragonflybsd.org}
  
  \freeos{DesktopBSD}{1.6}{BSD}{Unix-like}{Monolitico modulare}{x86}{www.desktopbsd.net}
  
  \freeos{PC-BSD}{7.0}{BSD}{Unix-like}{Monolitico modulare}{x86}{www.pcbsd.org}
  
  \freeos{Hurd}{CVS}{GPL}{Unix-like}{Microkernel}{x86}{www.gnu.org/software/hurd/}
  
  \freeos{FreeDOS}{1.0}{GPL}{DOS}{Monolitico}{i386}{www.freedos.org}
  
  \freeos{Plan9}{quarta edizione}{Lucent Public License}{POSIX}{Ibrido}
  {alpha, arm, mips, powerpc, sparc, x86}{plan9.bell-labs.com/plan9/}
  
  \freeos{QNX}{6.3.2}{?}{POSIX}{Microkernel realtime}
  {arm, mips, powerpc, superH, intel 8088, x86}{www.qnx.com}
  
  \freeos{Syllable}{0.6.5}{GPL}{POSIX}{Ibrido}{i386}{www.syllable.org}
  
  \freeos{JNode}{0.2.7}{LGPL}{Java}{Esotico}{x86}{www.jnode.org}
  
  \freeos{KolibriOS}{0.7.1.0}{GPL}{Esotico}{Monolitico}{i386}{www.kolibrios.org}
  
  \freeos{Darwin}{9.5.0}{Apple Public Source License}{Unix-like}{Ibrido}{arm, powerpc, x86}
  {developer.apple.com/opensource/}
 
 
  \begin{frame}
  \frametitle{Fine}
  Happy hacking!
  \end{frame}
  
\end{document}

