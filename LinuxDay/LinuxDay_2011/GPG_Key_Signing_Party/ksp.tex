
\documentclass{beamer}

\usepackage{graphicx}
\usepackage[italian]{babel}
\usepackage[utf8]{inputenc}
\usepackage{tikz}

\usetikzlibrary{shapes.geometric}

\usetheme{Boadilla}

\title{Key Signing Party}
\subtitle{Cos'è, a cosa serve e come funziona}
\author{Giovanni Mascellani}
\date[Linux Day 2011]{sabato 22 ottobre 2011 -- Linux Day 2011}
\institute[GULP]{Gruppo Utenti Linux -- Pisa}

\begin{document}

\begin{frame}
 \maketitle
\end{frame}

\begin{frame}{Gli scopi}
 \begin{description}
  \item[Crittografia] Alterare un documento in modo che solo determinate persone
   siano in grado di leggerlo.
  \item[Firma digitale] Aggiungere ad un documento alcuni dati che certificano
   con certezza chi è il suo autore e garantiscono che il documento stesso non
   è stato alterato durante il trasporto.
  \end{description}

  Oggi esistono tecnologie, ritenute sufficientemente sicure, che permettono di
  raggiungere questi due obiettivi.

  \pause

  \begin{alertblock}{Una catena è resistente quanto il suo anello più debole}
    Con la tecnologia di oggi gli anelli deboli non sono quelli centrali, ma
    gli estremi, ossia gli utenti che utilizzano impropriamente le tecnologie.
  \end{alertblock}
\end{frame}

\begin{frame}{Le chiavi}
 Una chiave di un meccanismo di crittografia a chiave pubblica è
 composta da due parti.

 \begin{description}
  \item[Chiave pubblica] Permette la cifratura di un documento e la verifica di
   una firma digitale. Può essere liberamente divulgata, in quanto si ritiene
   che non sia computazionalmente possibile ottenere la chiave privata a partire
   da quella pubblica.
  \item[Chiave privata] Permette la decifratura di un documento e la creazione
   di una firma digitale. Deve essere mantenuta rigorosamente segreta. Ottenere
   la chiave pubblica da quella privata è molto semplice.
 \end{description}

 \pause

 \begin{alertblock}{Problema di sicurezza}
   Chiunque può creare una chiave con un qualsiasi nome scritto sopra. Dunque
   è necessario verificare che il nome scritto su una chiave sia quello giusto
   prima di utilizzarla.
 \end{alertblock}
\end{frame}

\begin{frame}{Come risolvere il problema?}

  Per scrivere un messaggio segreto a Tizio devo verificare che la chiave con
  scritto Tizio sia veramente di Tizio.

  \only<1>{\begin{block}{Idea!\uncover<2->{?!?? (ma è fattibile? Se Tizio sta in Australia?)}}}
  \only<2->{\begin{alertblock}{Idea?!?? (ma è fattibile? Se Tizio sta in Australia?)}}
    Posso incontrare Tizio di persona e controllare se la copia della
    chiave che ho io è giusta. Da quel punto in poi potrò scrivere
    messaggi segreti a Tizio ogni volta che voglio.
  \only<1>{\end{block}}
  \only<2->{\end{alertblock}}

  \uncover<3->{
    \begin{block}{Ideona!}
      Se un amico comune (Caio, \alert<4->{del quale mi fido}) verifica
      la chiave di Tizio e la firma digitalmente, e io verifico la
      chiave di Caio, allora so anche che la chiave di Tizio è quella
      giusta.
    \end{block}
  }

  \begin{center}
    
\begin{tikzpicture}

\tikzset{ampersand replacement=\&}
\tikzstyle{key}=[ellipse,draw=black,minimum width=1.2cm,minimum height=0.7cm]

\only<3-4>{\matrix[row sep=1cm,column sep=0.9cm] {
\node [key] (io) {Io}; \& \node [key] (caio) {Caio}; \& \node [key]  (tizio) {Tizio}; \\};
\draw [->,bend left] (io) to (caio);
\draw [->,bend left] (caio) to (tizio);}

\only<5>{\matrix[row sep=1cm,column sep=0.9cm] {
\node [key] (io) {Io}; \& \node [key] (caio) {Caio}; \& \node [key]  (sempronio) {Sempronio}; \& \node [key]  (tizio) {Tizio}; \\};
\draw [->,bend left] (io) to (caio);
\draw [->,bend left] (caio) to (sempronio);
\draw [->,bend left] (sempronio) to (tizio);}

\only<6>{\matrix[row sep=1cm,column sep=0.9cm] {
\node [key] (io) {Io}; \& \node [key] (caio) {Caio}; \& \node [key]  (sempronio) {Sempronio}; \& \node [key,draw=white]  (puntini) {...}; \& \node [key]  (tizio) {Tizio}; \\};
\draw [->,bend left] (io) to (caio);
\draw [->,bend left] (caio) to (sempronio);
\draw [->,bend left] (sempronio) to (puntini);
\draw [->,bend left] (puntini) to (tizio);}

\pgfresetboundingbox
\path [use as bounding box] (-6,-0.5) rectangle (6,0.7);

\end{tikzpicture}

  \end{center}

\end{frame}

\begin{frame}{Il Web of Trust}

  Se ognuno firma digitalmente le chiavi che controllare essere
  corrette si forma una rete di fiducia, il Web of Trust. Chi firma
  una chiave asserisce di avere verificato che il nome scritto su di
  essa è corretto. \textbf{È quello che faremo anche noi tra poco.}

  \pause

  \begin{block}{Validità e fiducia}
    Una chiave può essere:
    \begin{description}
      \item [valida] \emph{``Sono sicuro che il proprietario
        dichiarato è corretto''} (o perché l'ho verificato di persona
        o perché mi fido di una firma sulla chiave)
      \item [invalida] \emph{``Non ho garanzia che il proprietario
        dichiarato sia corretto''}
    \end{description}

    Se la chiave è valida posso decidere di:
    \begin{description}
      \item [fidarmi] \emph{``Considero degne di fiducia le firme
        fatte con questa chiave''}
      \item [non fidarmi] \emph{``Non do fiducia alle firme di questa
        chiave, perché ritengo che il proprietario sia inaffidabile''}
    \end{description}
  \end{block}

\end{frame}

\begin{frame}[fragile]{Verificare una chiave}

  Ad ogni chiave è associato un fingerprint (``impronta digitale''):
  se due chiavi hanno lo stesso fingerprint, allora sono (quasi)
  sicuramente uguali.

  {\scriptsize
\begin{verbatim}
$ gpg --fingerprint D9AB457E
pub   4096R/D9AB457E 2010-02-24 [expires: 2012-04-22]
      Key fingerprint = 82D1 19A8 40C6 EFCA 6F5A  F945 9EDC C991 D9AB 457E
uid                  Giovanni Mascellani <mascellani@poisson.phc.unipi.it>
[...]
\end{verbatim}}

\pause

Chi vuole verificare questa chiave deve:
\begin{itemize}
\item Controllare che il mio vero nome sia Giovanni Mascellani (o
  perché mi conosce o vedendo un mio documento di identità);
\item Controllare che l'email \verb-mascellani@poisson.phc.unipi.it-
  sia mia (lo stesso controllo va fatto anche per le altre email);
\item Controllare che il fingerprint della mia chiave conincida con
  quello che ha lui.
\end{itemize}

\textbf{Fatte queste verifiche può firmare la mia chiave.}

\end{frame}

\end{document}

