\documentclass{beamer}

\beamertemplateshadingbackground{red!10}{blue!10} %10 10

\usetheme{Warsaw}

\beamertemplatetransparentcovereddynamic

\usepackage[utf8x]{inputenc}
\usepackage[T1]{fontenc}

\usepackage[italian]{babel}

\title{Scacco matto col Pinguino!}
\author{Gabriele ``LightKnight'' Stilli}
\institute{G.U.L.P. -- Gruppo Utenti Linux Pisa}
\date{Linux Day, Pisa, 22/10/2011}

\begin{document}
\frame{\titlepage}

\frame{\tableofcontents}
\section{Introduzione}

\begin{frame}\frametitle{Com'eravamo, come siamo}
Fino a poco tempo fa, l'informatica scacchistica era appannaggio di
poche aziende proprietarie; le alternative libere erano poche e poco
sviluppate.

Recentemente sono nati alcuni nuovi programmi liberi che possono
competere con gli equivalenti proprietari; inoltre, alcuni dei
programmi esistenti sono migliorati.
\end{frame}

\section{Database}

\begin{frame}\frametitle{SCID}

SCID (Shane's Chess Information Database) è un database scacchistico
con funzioni di:
\begin{itemize}
\item inserimento partite;
\item ricerca per intestazione, posizione, struttura ecc.;
\item esportazione grafica (\LaTeX, PDF, HTML) e PGN;
\item produzione di rapporti di apertura;
\item allenamento tattico;
\item analisi con motori.
\end{itemize}

\end{frame}

\begin{frame}\frametitle{Derivati di SCID}

Negli anni, sono nati alcuni fork di SCID volti a migliorare aspetti
specifici del programma che alcuni ritenevano insoddisfacenti.

I più promettenti, seppur recenti, sono SCIDvsPC e SCIDB.

\end{frame}

\section{Motori}

\begin{frame}\frametitle{Panoramica}

Negli ultimi anni, oltre ai motori tradizionali (es. GNUchess) sono
nati alcuni motori di forza quasi pari alle alternative commerciali:

\begin{itemize}
\item Fruit (e il suo fork TogaII);
\item Hoichess;
\item Glaurung e il suo erede Stockfish, probabilmente il più forte
motore libero esistente oggi.
\end{itemize}

\end{frame}

\section{Interfacce}

\begin{frame}\frametitle{Per server e motori}

Le interfacce grafiche sono utili per interagire con i motori e
giocare sui server online e contro altri giocatori.

Le interfacce più diffuse e sviluppate sono:

\begin{itemize}
\item eboard;
\item raptor;
\item pychess;
\item xboard.
\end{itemize}

\end{frame}

\section{Varie}

\begin{frame}\frametitle{Tornei, web e altro}

\begin{itemize}
\item pgn4web (pubblicazione partite sul web);
\item JavaPairing (gestione tornei);
\item dgtnix/dgtdrv (collegamento con scacchiere elettroniche).
\end{itemize}

\end{frame}

\begin{frame}\frametitle{Copyright (o Copyleft)}
Quest'opera è stata rilasciata sotto la licenza Creative Commons
Attribuzione-Condividi allo stesso modo 3.0 Unported. Per leggere una
copia della licenza visita il sito web
http://creativecommons.org/licenses/by-sa/3.0/ o spedisci una lettera
a Creative Commons, 171 Second Street, Suite 300, San Francisco,
California, 94105, USA.
\end{frame}

\end{document}
