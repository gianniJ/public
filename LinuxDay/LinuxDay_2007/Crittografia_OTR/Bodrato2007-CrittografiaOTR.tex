\documentclass{beamer}
\newcommand{\skipsignature}[1]{}
\skipsignature{
-----BEGIN PGP SIGNED MESSAGE-----
Hash: SHA1

% Questo sorgente LaTeX e` stato firmato con GnuPG,
% ad esclusione di un breve preambolo.
% Oltre alla parte firmata il documento originale contiene
% solo tre righe iniziali, riportate (commentate) di seguito:

% \documentclass{beamer}
% \newcommand{\skipsignature}[1]{}
% \skipsignature{

% Tutto il resto di questo sorgente e` firmato.
}
%
% Scritto da Marco Bodrato, ottobre 2007
% Pubblicato con liceza Creative Common 3.0 BY-NC-SA, disponibile presso
% http://creativecommons.org/licenses/by-nc-sa/3.0/
\usepackage{hyperref}
\usepackage{xcolor}
\usepackage{beamerthemesplit}

%\usetheme{Ilmenau}
%\usetheme{Berkley}
\usetheme{Copenhagen}

\newcommand{\grigio}[1]{\textcolor{white!60!black}{#1}}

% \title[Crittografia e partecipazione]{Crittografia: potenzialit\`a e limiti nella partecipazione digitale}
% \subtitle{Fiducia o trasparenza?}
% \author{Marco Bodrato}
% \institute[0xC1A000B0]{Universit\`a di Roma ``Tor Vergata''}
\title{Verba volant scripta manent.}
\subtitle{Quando, come e soprattutto {\bf quale} crittografia.}
\author[\href{http://marco.bodrato.it/}{Marco Bodrato}\hspace{1cm}(0xC1A000B0)]{Marco Bodrato}
%\institute[Tor-Vergata]{Universit\`a di Roma ``Tor Vergata''}
\date[LD 2007]{Linux Day - Pisa - 27 ottobre 2007}

\subject{Crittografia Off-the-Record}
\keywords{crittografia,rinnegabilit\`a,firma,riservatezza}

\begin{document}
\frame{\titlepage}

%\section*{Indice}
\frame{\tableofcontents}
\section{Modelli di sicurezza}
\subsection{Cosa vuole dire sicurezza?}
\frame{
  \frametitle{Una citazione}
  \framesubtitle{\ldots non prendiamoci sul serio \ldots}
  \begin{block}{Mohamed el Baradei\hfill\small Nobel per la pace 2005}
    Per molti popoli e nazioni la sicurezza resta una preoccupazione
    prioritaria. Ma in cosa essa consista, e quali siano le strategie
    per conseguirla, pu\`o variare molto.

    Per miliardi di persone sicurezza \`e la speranza di vedere
    ``garantiti'' i propri bisogni fondamentali: cibo, acqua, un tetto,
    l'assistenza sanitaria.

    Per altri, sicurezza \`e la speranza di vedere ``garantiti'' altri
    diritti umani fondamentali, quali la libert\`a d'espressione e
    dissenso, \ldots%% e in primo luogo la libert\`a dall'oppressione.
%%
%%    Per altri ancora, \`e la speranza di emanciparsi da un'occupazione
%%    straniera.
    \begin{center}
      \href{http://ln.bodrato.it/Baradei-discorso}{$\cdots$}
    \end{center}
    
    \grigio{\hfill Universit\`a di Firenze, 5 ottobre 2007}
  \end{block}
}
\frame{
  \frametitle{Sicurezza nella comunicazione}
  \begin{block}{}
    Per dire che un canale di comunicazione \`e sicuro ci aspettiamo
    almeno alcune garanzie:
    \begin{itemize}
    \item l'identit\`a dell'interlocutore
    \item che non ci siano altri ad ascoltare
    \item messaggi non distorti
    \end{itemize}
  \end{block}
  \vspace{1cm}

  Per ogni forma di comunicazione tradizionale, {\it conosciamo il
    livello di sicurezza} e adeguiamo il tono e i contenuti della
  discussione.

  Non tutti i canali devono essere sicuri, ma vogliamo saperlo.
}
\subsection{Quel che ci aspettiamo dalla posta}
\frame{
  \frametitle{Posta e posta elettronica}
  \begin{block}{Busta chiusa}
    {\it Nessuno} pu\`o leggere il contenuto senza manomettere
    l'involucro. La grafia ci conforta sull'identit\`a del mittente.
  \end{block}
%%  \pause
  \begin{block}{Posta elettronica}
    \`E paragonabile alle cartoline\ldots\\
    \ldots scritte a macchina
  \end{block}

  Falsificare messaggi di posta elettronica \`e tecnicamente banale,
  solo una questione di volont\`a.

  Servono firma elettronica e cifratura, per questo si usano
  protocolli come OpenPGP o SSL.
}
\subsection{\ldots e da una chiacchierata tra amici}
\frame{
  \frametitle{Chiacchiere e chat}
  \begin{block}{Discussione a casa di amici}
    Vediamo i nostri interlocutori e possiamo immaginare che nessuno
    abbia un registratore in funzione\ldots
  \end{block}
%%  \pause
  \begin{block}{Chat via internet}
    Identit\`a garantita in modo molto debole\ldots\\
    \ldots quasi certezza che {\it il registratore} sia in funzione.
  \end{block}

  In ogni caso la trascrizione elettronica \`e un testo facilmente
  falsificabile.

  Qui serve la crittografia {\it Off-the-Record}, ``immune alla
  registrazione''.
}
\section{Vari progammi/paradigmi per la crittografia}
\subsection{HTTPS, SSL\ldots}
\begin{frame}
  \frametitle{Canali sicuri}
  \framesubtitle{https://\ldots o SSL}
  Esistono degli standard abbastanza consolidati per creare
  connessioni {\it sicure}.

  \begin{block}{Le domande da porsi\ldots}
    \begin{itemize}
    \item Chi garantisce questa sicurezza? (2x)
    \item Che tipo di sicurezza \`e?
    \item \`E adeguata ai {\it miei} bisogni di sicurezza?
    \end{itemize}
  \end{block}

  Ad esempio l'uso di HTTPS o SSL per leggere/spedire posta o per
  partecipare a chat. Sono protocolli adeguati?

\end{frame}
\subsection{GPG o OpenPGP}
\frame{
  \frametitle{Per la posta elettronica\ldots}

  Per usare in modo sicuro la posta elettronica conviene crearsi una
  chiave GPG. Ed imparare ad usare un poco di crittografia\ldots

  \begin{block}{Vantaggi}
    \begin{itemize}
    \item Se usate la cifratura, solo i destinatari potranno decifrare
      il messaggio.
    \item<alert@2> Se usate la firma, chiunque, in ogni momento, potr\`a
      verificare che siete stati voi a scrivere quello che avete
      scritto.
    \end{itemize}
  \end{block}
  \pause
  \begin{alertblock}{Svantaggi}
    \begin{itemize}
    \item Se usate la firma, chiunque, in ogni momento, potr\`a
      verificare che siete stati voi a scrivere quello che avete
      scritto.
    \end{itemize}
  \end{alertblock}
  
  Ci son caratteristiche desiderabili in alcune occasioni e
  {\bf non} in altre.

}
\subsection{E cosa succede su una chat?}
\frame{
  \frametitle{Cosa succede se uso i paradigmi precedenti su una chat?}
  Si possono firmare e cifrare tutti i messaggi.

  \begin{block}{Installazione su Debian di un plug-in per pidgin\ldots}
    \tt \# \  apt-get install pidgin-encryption
  \end{block}

  In questo modo otterr\`o le garanzie di identit\`a e riservatezza di
  prima, ma sono solo questi gli effetti?

  \begin{alertblock}{Questo significa firmare ogni messaggio}
    Siete davvero disposti a mettere per scritto e firmare ogni vostra
    chiacchiera?
  \end{alertblock}

  Bisogna inventarsi qualcosa d'altro\ldots
}
\section{Crittografia Off-the-Record}
\subsection{Cosa ci garantisce}
\frame{
  \frametitle{Cosa ci garantisce?}
  \framesubtitle{Quello che ci aspettiamo da una chiacchierata privata\ldots}
  \begin{itemize}[<+-| alert@+>]
    \item Riservatezza: nessun'altro pu\`o leggere i nostri messaggi.
    \item Autenticazione: garantita l'identit\`a del nostro
      interlocutore e l'integrit\`a del messaggio.
    \item {\it Perfect forward secrecy}: una compromissione della
      chiave privata non fornisce informazioni sui messaggi passati.
    \item Rinnegabilit\`a: i messaggi {\bf non} contengono una firma
      digitale verificabile da terzi. \`E facile falsificare la
      trascrizione della conversazione.
  \end{itemize}
}
\subsection{Programmi che la implementano}
\frame{
  \frametitle{Programmi liberi}
  Per ora solo alcuni programmi per chat via rete:
  \begin{itemize}
  \item climm (ex mICQ),
  \item kopete (via plug-in),
  \item<alert@1> pidgin (via plug-in).
  \end{itemize}
  
  Quest'ultimo \`e l'implementazione migliore che potete sperare, visto
  che \`e curata ed aggiornata degli inventori del protocollo.

  \begin{block}{Installazione su Debian}
    \tt \# \  apt-get install pidgin-otr
  \end{block}
}

\frame{
  \frametitle{Configurazione ed uso di pidgin-otr}
  \begin{block}{La configurazione e l'uso in pochi passi\ldots}
    \begin{enumerate}
    \item Tools$\to$Plugins$\to$Off-the-Record attivate il plugin
    \item Durante la chat a 2: il tasto ``OTR'' in basso inizia la sessione protetta
    \item Mouse-destro sul tasto ``OTR''$\to$Authenticate per verificare l'identit\`a.
    \end{enumerate}
  \end{block}
}
\subsection*{Domande?}
\frame{
  \frametitle{Grazie}
  \framesubtitle{Domande?}

  \begin{center}
    Grazie per l'attenzione!
    
    \vspace{8mm}

    \alert{\Huge Domande?}

    \vspace{8mm}

    Per informazioni aggiornate:\\\url{http://www.cypherpunks.ca/otr/}

    \vspace{8mm}

    \footnotesize \grigio{La presentazione \`e disponibile via web:
      \href{http://bodrato.it/presentazioni/\#LD2007}{http://bodrato.it/presentazioni/\#LD2007},\\
      Rilasciata con
      \href{http://creativecommons.org/licenses/by-nc-sa/3.0/}{licenza
        CreativeCommons BY-NC-SA.
        \includegraphics[width=1cm]{cc.png}}}
  \end{center}
}
\frame{
  \frametitle{E per la posta?}

  \`E teoricamente possibile un protocollo del genere anche per la
  posta, ma \`e pi\'u complicato. Per ora non esistono implementazioni.

  Tra i problemi:
  \begin{itemize}
  \item la interattivit\`a necessaria per attivare il protocollo,
  \item ritardo tra un messaggio e un altro.
  \end{itemize}
}
\frame{
  \frametitle{E per la voce?}

  A parte il fatto che potrebbe essere illegale cifrare la voce, e a
  quanto mi risulta in Italia lo \`e, tecnicamente \`e fattibile, ma\ldots

  \begin{block}{La solita domanda:}
    La sicurezza data dal protocollo OTR \`e adeguata al caso della
    trasmissione vocale?
  \end{block}

  \begin{alertblock}{NO!}
    Come possiamo garantire la rinnegabilit\`a della voce?

    La registrazione analogica di una frase detta col nostro accento,
    timbro, pronuncia, cadenza\ldots pu\`o risultare incontestabile.
  \end{alertblock}

  Le parole dette viaggiano in volo, incontrollabili; ci\`o che \`e
  scritto rimane, passibile di correzione. Cio che \`e per l'uno non \`e
  per l'altro.

}
\end{document}
-----BEGIN PGP SIGNATURE-----
Version: GnuPG v1.4.6 (GNU/Linux)

iEYEARECAAYFAkcl/nIACgkQ5MyCX967pLpn3ACdElecghpM5p197+29yCp69kqW
FAQAmwZ4B4RFaKR9sojmTBN+tmQANcor
=Kj3d
-----END PGP SIGNATURE-----
