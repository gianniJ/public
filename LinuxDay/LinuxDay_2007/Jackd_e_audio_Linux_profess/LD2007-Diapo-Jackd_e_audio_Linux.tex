\documentclass{beamer}
    \usepackage[utf8]{inputenc} % così possiamo usare le lettere accentate
    \usepackage[italian]{babel}

\usepackage{beamerthemesplit}

\title{Jackd - Audio Linux professionale}
\author{Pietro Battiston}
\date{\today}

\begin{document}

\frame{\titlepage}

\section[Outline]{}
\frame{
\tableofcontents}

\section{Uno studio di registrazione con Linux: gli ingredienti}
\subsection{Kernel a bassa latenza}
\frame{Sono kernel ottimizzati per le applicazioni in \emph{real time}, in cui non ci interessa solo \emph{cosa} un programma fa, ma \emph{quando} lo fa.\\
Per garantire che i programmi real time riescano ad ottenere il controllo della CPU nei momenti esatti in cui ne hanno bisogno, si modifica le procedure di scheduling (\emph{preemption}).}

\frame{
  \frametitle{Pacchetti}
Pacchetti:
\begin{itemize}
 \item Debian: pacchetto \emph{rtlinux}
 \item Ubuntu: pacchetto \emph{linux-image-rt}
\end{itemize}
}
\subsection{Jack e accessori}
\frame{Jack (\emph{JACK Audio Connection Kit}) è un server audio versatile ed efficiente.\\
Si interfaccia con ALSA e vari altri sistemi audio.\\
Gira come demone: \emph{jackd}.\\
Per avviarlo e configurarlo, esiste \emph{qjackctl}.\\}

\subsection{Ardour}
\frame{Ardour (www.ardour.org) è un potente registratore ed editor multitraccia, basato su jack.\\
È indirizzato (anche) ad un utilizzo professionale, come uno studio di registrazione.\\
È importante quindi la buona integrazione con schede audio, consolle...}
\subsection{Gadget}
\frame{
\frametitle{I gadget}
Sono moltissimi i programmi che si appoggiano su Jack: vedremo \emph{JACK rack}, \emph{Hydrogen}, \emph{Meterbridge}...}

\frame{
\frametitle{JACK rack}
Utilizza i plugin \emph{LADSPA} per fornire effetti in tempo reale.
}
\end{document}